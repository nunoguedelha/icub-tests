\hypertarget{writing-and-running_writing-new-tests}{}\section{Writing a new test}\label{writing-and-running_writing-new-tests}
Create a folder with the name of your test case in the {\ttfamily icub-\/tests/src/} folder to keep your test codes\-:

``` \$ mkdir icub-\/tests/src/example-\/test ```

Create a child test class inherited from the {\ttfamily Yarp\-Test\-Case}\-:

```c++ \#ifndef {\itshape E\-X\-A\-M\-P\-L\-E\-\_\-\-T\-E\-S\-T\-\_\-\-H} \#define {\itshape E\-X\-A\-M\-P\-L\-E\-\_\-\-T\-E\-S\-T\-\_\-\-H}

\#include $<$Yarp\-Test\-Case.\-h$>$

class \hyperlink{classExampleTest}{Example\-Test} \-: public Yarp\-Test\-Case \{ public\-: \hyperlink{classExampleTest}{Example\-Test()}; virtual $\sim$\-Example\-Test();

virtual bool setup(yarp\-::os\-::\-Property\& property);

virtual void tear\-Down();

virtual void run(); \};

\#endif //\-\_\-\-E\-X\-A\-M\-P\-L\-E\-\_\-\-T\-E\-S\-T\-\_\-\-H ```

Implement the test case\-:

```c++ \#include $<$Plugin.\-h$>$ \#include \char`\"{}\-Example\-Test.\-h\char`\"{}

using namespace std; using namespace R\-T\-F; using namespace yarp\-::os;

prepare the plugin P\-R\-E\-P\-A\-R\-E\-\_\-\-P\-L\-U\-G\-I\-N(\-Example\-Test)

Example\-Test\-::\-Example\-Test() \-: Yarp\-Test\-Case(\char`\"{}\-Example\-Test\char`\"{}) \{ \}

Example\-Test\-::$\sim$\-Example\-Test() \{ \}

bool Example\-Test\-::setup(yarp\-::os\-::\-Property \&property) \{

initialization goes here ... updating the test name if(property.\-check(\char`\"{}name\char`\"{})) set\-Name(property.\-find(\char`\"{}name\char`\"{}).as\-String());

string example = property.\-check(\char`\"{}example\char`\"{}, Value(\char`\"{}default value\char`\"{})).as\-String();

R\-T\-F\-\_\-\-T\-E\-S\-T\-\_\-\-R\-E\-P\-O\-R\-T(Asserter\-::format(\char`\"{}\-Use '\%s' for the example param!\char`\"{}, example.\-c\-\_\-str())); return true; \}

void Example\-Test\-::tear\-Down() \{ finalization goes here ... \}

void Example\-Test\-::run() \{ \begin{DoxyVerb}int a = 5; int b = 3;
RTF_TEST_REPORT("testing a < b");
RTF_TEST_CHECK(a<b, Asserter::format("%d is not smaller than %d.", a, b));
RTF_TEST_REPORT("testing a > b");
RTF_TEST_CHECK(a>b, Asserter::format("%d is not smaller than %d.", a, b));
RTF_TEST_REPORT("testing a == b");
RTF_TEST_CHECK(a==b, Asserter::format("%d is not smaller than %d.", a, b));
\end{DoxyVerb}
 add more ... \} ```

Notice\-: The {\ttfamily R\-T\-F\-\_\-\-T\-E\-S\-T\-\_\-\-C\-H\-E\-C\-K}, {\ttfamily R\-T\-F\-\_\-\-T\-E\-S\-T\-\_\-\-R\-E\-P\-O\-R\-T} do N\-O\-T threw any exception and are used to add failure or report messages to the result collector. Instead, all the macros which include {\ttfamily \-\_\-\-A\-S\-S\-E\-R\-T\-\_\-} within their names (e.\-g., {\ttfamily R\-T\-F\-\_\-\-A\-S\-S\-E\-R\-T\-\_\-\-F\-A\-I\-L}) throw exceptions which prevent only the current test case (Not the whole test suite) of being proceed. The error/failure messages thrown by the exceptions are caught. (See \href{http://robotology.github.io/robot-testing/documentation/TestAssert_8h.html}{\tt $\ast$\-Basic Assertion macros$\ast$}).

The report/assertion macros store the source line number where the check/report or assertion happen. To see them, you can run the test case or suit with {\ttfamily -\/-\/detail} parameter using the {\ttfamily testrunner} (See \href{http://robotology.github.io/robot-testing/documentation/testrunner.html}{\tt $\ast$\-Running test case plug-\/ins using testrunner$\ast$}).

Create a cmake file to build the plug-\/in\-:

```cmake cmake\-\_\-minimum\-\_\-required(V\-E\-R\-S\-I\-O\-N 2.\-8.\-9)

\section*{set the project name}

set(\-P\-R\-O\-J\-E\-C\-T\-N\-A\-M\-E Example\-Test) project(\$\{P\-R\-O\-J\-E\-C\-T\-N\-A\-M\-E\})

\section*{add the required cmake packages}

find\-\_\-package(\-R\-T\-F) find\-\_\-package(\-R\-T\-F C\-O\-M\-P\-O\-N\-E\-N\-T\-S D\-L\-L) find\-\_\-package(\-Y\-A\-R\-P)

\section*{add include directories}

include\-\_\-directories(\$\{C\-M\-A\-K\-E\-\_\-\-S\-O\-U\-R\-C\-E\-\_\-\-D\-I\-R\} \$\{R\-T\-F\-\_\-\-I\-N\-C\-L\-U\-D\-E\-\_\-\-D\-I\-R\-S\} \$\{Y\-A\-R\-P\-\_\-\-I\-N\-C\-L\-U\-D\-E\-\_\-\-D\-I\-R\-S\} \$\{Y\-A\-R\-P\-\_\-\-H\-E\-L\-P\-E\-R\-S\-\_\-\-I\-N\-C\-L\-U\-D\-E\-\_\-\-D\-I\-R\})

\section*{add required libraries}

link\-\_\-libraries(\$\{R\-T\-F\-\_\-\-L\-I\-B\-R\-A\-R\-I\-E\-S\} \$\{Y\-A\-R\-P\-\_\-\-L\-I\-B\-R\-A\-R\-I\-E\-S\})

\section*{add the source codes to build the plugin library}

add\-\_\-library(\$\{P\-R\-O\-J\-E\-C\-T\-N\-A\-M\-E\} M\-O\-D\-U\-L\-E \hyperlink{ExampleTest_8h_source}{Example\-Test.\-h} \hyperlink{ExampleTest_8cpp_source}{Example\-Test.\-cpp})

\section*{set the installation options}

install(T\-A\-R\-G\-E\-T\-S \$\{P\-R\-O\-J\-E\-C\-T\-N\-A\-M\-E\} E\-X\-P\-O\-R\-T \$\{P\-R\-O\-J\-E\-C\-T\-N\-A\-M\-E\} C\-O\-M\-P\-O\-N\-E\-N\-T runtime L\-I\-B\-R\-A\-R\-Y D\-E\-S\-T\-I\-N\-A\-T\-I\-O\-N lib) ```

Call your cmake file from the {\ttfamily icub-\/test/\-C\-Make\-Lists.\-txt} to build it along with the other other test plugins. To do that, adds the following line to the {\ttfamily icub-\/test/\-C\-Make\-Lists.\-txt}

``` \section*{Build example test}

add\-\_\-subdirectory(src/example-\/test) ```

Please check the {\ttfamily icub-\/tests/example} folder for a template for developing tests for the i\-Cub.\hypertarget{writing-and-running_running_single_test_case}{}\section{Running a single test case}\label{writing-and-running_running_single_test_case}
As it is documented here (\href{http://robotology.github.io/robot-testing/documentation/testrunner.html}{\tt $\ast$\-Running test case plug-\/ins using testrunner$\ast$}) you can run a single test case or run it with the other tests using a test suite. For example, to run a single test case\-:

``` testrunner --verbose --test plugins/\-Example\-Test.\-so --param \char`\"{}-\/-\/name My\-Example\-Test\char`\"{} ```

Notice that this test require the {\ttfamily yarpserver} to be running and it contains tests that are programmed to succeed and some that are programmed to fail.

or to run the i\-Cub\-Sim camera test whith the test configuration file\-:

``` testrunner --verbose --test plugins/\-Camera\-Test.\-so --param \char`\"{}-\/-\/from camera\-\_\-right.\-ini\char`\"{} --environment \char`\"{}-\/-\/robotname icub\-Sim\char`\"{} ```

This runs the icub\-Sim right-\/camera test with the parameters specified in the {\ttfamily right\-\_\-camera.\-ini} which can be found in {\ttfamily icub-\/tests/suits/contexts/icub\-Sim} folder. This test assumes you are running {\ttfamily yarpserver} and the i\-Cub simulator (i.\-e. {\ttfamily i\-Cub\-\_\-\-S\-I\-M}).

Notice that the environment parameter {\ttfamily -\/-\/robotname icub\-Sim} is used to locate the correct context (for this examples is {\ttfamily icub\-Sim}) and also to update the variables loaded from the {\ttfamily right\-\_\-camera.\-ini} file.\hypertarget{writing-and-running_running_multiple_tests}{}\section{Running multiple tests using a test suite}\label{writing-and-running_running_multiple_tests}
You can update one of the existing suite X\-M\-L files to add your test case plug-\/in and its parameters or create a new test suite which keeps all the relevant test cases. For example the {\ttfamily basic-\/icub\-Sim.\-xml} test suite keeps the basic tests for cameras and motors\-:

```xml $<$?xml version=\char`\"{}1.\-0\char`\"{} encoding=\char`\"{}\-U\-T\-F-\/8\char`\"{}?$>$

$<$suit name=\char`\"{}\-Basic Tests Suite\char`\"{}$>$ Testing robot's basic features $<$environment$>$--robotname icub\-Sim$<$/environment$>$ $<$fixture param=\char`\"{}-\/-\/fixture icubsim-\/fixture.\-xml\char`\"{}$>$ yarpmanager $<$/fixture$>$

\begin{DoxyVerb}<test type="dll" param="--from right_camera.ini"> CameraTest </test>
<test type="dll" param="--from left_camera.ini"> CameraTest </test> 


<test type="dll" param="--from test_right_arm.ini"> MotorTest </test>
<test type="dll" param="--from test_left_arm.ini"> MotorTest </test>
\end{DoxyVerb}
 $<$/suit$>$

```

Then you can run all the test cases from the test suite\-:

``` testrunner --verbose --suit icub-\/tests/suits/basics-\/icub\-Sim.\-xml ```

The {\ttfamily testrunner}, first, launches the i\-Cub simulator and then runs all the tests one after each other. After running all the test cases, the {\ttfamily tesrunner} stop the simulator. If the i\-Cub simulator crashes during the test run, the {\ttfamily testrunner} re-\/launchs it and continues running the remaining tests.

How {\ttfamily testrunner} knows that it should launch the i\-Cub simulator before running the tests? Well, this is indicated by {\ttfamily $<$fixture param=\char`\"{}-\/-\/fixture icubsim-\/fixture.\-xml\char`\"{}$>$ yarpmanager $<$/fixture$>$}. The {\ttfamily testrunner} uses the {\ttfamily yarpmanager} fixture plug-\/in to launch the modules which are listed in the {\ttfamily icubsim-\/fixture.\-xml}. Notice that all the fixture files should be located in the {\ttfamily icub-\/tests/suits/fixtures} folder. 